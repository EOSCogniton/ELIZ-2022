\documentclass{EPSA-rap-template}


\type{Guide}

\titresize{\large} % ne pas hésiter a changer la taille :
%\normalsize
%\large
%\Large
%\LARGE
%\huge
%\Huge
%\HUGE

\titre{ Explication des modèles de simulation proteus}
\titresh{Model Explanation} % Titre réduit pour les en-tête et bas de page

\departement{BASTIE et CHAIPE}
\departementsh{Elec}% Département réduit pour les en-tête et bas de page

\auteurs{Eymeric \textbf{Chauchat}}

\version{V1}

\versionnement{
\ver{V1}{22 Octobre 2022}{ECT}{Rédaction initiale.}{1} 
}



\setuppack



\begin{document}

\fairepagedegarde
\newpage
\tableofcontents

\newpage

\section{Introduction}

Ce présent document vise a fournir un guide sur l'installation de tout les composants dans un simulateur (proteus, LTspice ou autre).

\section{Recherche des modèles de simulation}

\subsection{Résistance, condensateur et trimmers}

Hormis les empreintes qui doivent être correctement rentré, il n'y a pas de modèle spécifique.

\subsection{Timer 555}

On utilise le modèle Spice du NE555 (suffisant pour nos applications)

\subsection{Comparateur}

Texas Instrument propose les modèles Spice de leur composant : 
\begin{itemize}
\item TS391
\item LM393B
\end{itemize}

\subsection{Alimentation}

Il n'y a pas de besoin de modèle de simulation, on considère qu'elles sont idéales

\subsection{Porte logique}

Chaque porte logique possède son modèle de simulation trouvable sur Texas Instrument.

\subsection{Solid State}

Un modèle de Relays suffit

\subsection{Bascule D}

Le modèle spice est celui de la bascule D SN74HCS72

\subsection{Diode}

On utilise le modèle Spice de la diode RFN1VWM2STF

\subsection{Mosfet}

Pour le mosfet P, il y a un modèle spice pour Si2365EDS
Pour le mosfet N, on a choisit un modèle spice légerement différent

\end{document}