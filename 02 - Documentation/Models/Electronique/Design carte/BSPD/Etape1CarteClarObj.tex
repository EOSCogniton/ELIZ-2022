\documentclass{EPSA-rap-template}

\usepackage{listings}

\newcommand{\TitreCarte}[1]{\renewcommand{\TitreCarte}{#1}}
\newcommand{\SectionPrimaire}[1]{\renewcommand{\SectionPrimaire}{#1}}
\newcommand{\Vehicule}[1]{\renewcommand{\Vehicule}{#1}}
\newcommand{\OrRule}[1]{\renewcommand{\OrRule}{#1}}
\newcommand{\entree}[1]{\renewcommand{\entree}{#1}}
\newcommand{\sortie}[1]{\renewcommand{\sortie}{#1}}
\newcommand{\descDet}[1]{\renewcommand{\descDet}{#1}}
\newcommand{\lienSchema}[1]{\renewcommand{\lienSchema}{#1}}
\newcommand{\verRegle}[1]{\renewcommand{\verRegle}{#1}}
\newcommand{\secAnn}[1]{\renewcommand{\secAnn}{#1}}
\newcommand{\specAn}[1]{\renewcommand{\specAn}{#1}}
\newcommand{\comLog}[1]{\renewcommand{\comLog}{#1}}
\newcommand{\lienSchemaRl}[1]{\renewcommand{\lienSchemaRl}{#1}}
\newcommand{\lienSchemaPl}[1]{\renewcommand{\lienSchemaPl}{#1}}
\newcommand{\listecomp}[1]{\renewcommand{\listecomp}{#1}}
\newcommand{\lienData}[1]{\renewcommand{\lienData}{#1}}

\usepackage{CarteElec}


\type{Clarification Objectif Carte Electroniques}

\titre{\TitreCarte}

\departement{BASTIE}
\version{v0.1}




\setuppack

\begin{document}

\fairemarges
\fairepagedegarde

\newpage

\tableofcontents

\section{Introduction}

Le présent document va résumer le fonctionnement la carte \TitreCarte \ présent dans le réglement à la section \SectionPrimaire. Cette rédaction s'est faite sous la direction du véhicule \Vehicule \ et sous la version des régles suivantes : \verRegle 

\section{Description du fonctionnement}

La logique de cette carte obéit aux règles suivantes : 

{
\scriptsize	
\lstinputlisting[breaklines,mathescape=true]{Regle.txt}

}

\bigskip

De ces règles nous pouvons sortir une liste d'entrée sortie que devra posséder notre carte :

\textbf{Entrée :}

\begin{itemize}

\entree

\end{itemize}

\textbf{Sortie :}

\begin{itemize}

\sortie

\end{itemize}

\section{Description détaillé du fonctionnement de la carte}

Le fonctionnement détaillé de la carte \TitreCarte \ est résumé ci-dessous : 

\descDet

\section{Schéma Logique}

La carte \TitreCarte , devra suivre le schéma logique Figure \ref{fig:sch_log}. Ce schéma pose les bases pour la conception détaillé de la carte. 

\begin{figure}[hbt!]
\centering
\includegraphics[width=\textwidth]{\lienSchema}
\caption{Schéma logique de la carte \TitreCarte}
\label{fig:sch_log}
\end{figure}

\comLog

\section{Réalisation de la carte \TitreCarte \ durant le projet \Vehicule}

\specAn

La carte de l'année actuelle possède un circuit réelle représenté Figure \ref{fig:sch_rl} et une optimisation d'espace Figure \ref{fig:sch_crly}. Toute les données concernant cette carte sont trouvables au lien suivant : \href{\lienData}{Itération \Vehicule \ de la carte \TitreCarte }

\begin{figure}[hbt!]
\centering
\includegraphics[width=\textwidth]{\lienSchemaRl}
\caption{Schéma Réelle de la carte \TitreCarte \ du projet \Vehicule}
\label{fig:sch_rl}
\end{figure}

\begin{figure}[hbt!]
\centering
\includegraphics[width=\textwidth]{\lienSchemaPl}
\caption{Schéma position de la carte \TitreCarte \ du projet \Vehicule}
\label{fig:sch_crly}
\end{figure}

La carte possède les composants suivants :

\begin{center}
{
\centering
\noindent
\renewcommand*{\arraystretch}{1.4}
\begin{tabular}{lll}

Pièce & Référence & Quantité \\
\hline
\listecomp

\end{tabular}
}
\end{center}

\secAnn

\end{document}